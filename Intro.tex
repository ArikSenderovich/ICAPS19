\label{sec:introduction}

Modern scheduling algorithms 
successfully solve
problems with thousands of
activities, complex temporal constraints,
and scarce resources. The main two prerequisites to 
solving scheduling problems are 
(1) a mathematical
model that describes the system
and (2) an algorithm that 
solves the problem based on the aforementioned model.

In deterministic scheduling,
Constraint Programming (CP) has been shown to 
be extremely effective in solving a wide range
of problems~\cite[Chapter 6]{Baptiste:2001},~\cite{lallouet2010learning}.  
However, creating CP
models  
often requires 
a high level
of expertise, as one must specify 
suitable variables, constraints and objective
functions that 
correspond to the underlying real-world problem~\cite{lallouet2010learning}. 

In this work, we address the challenge
of automating 
the process of modeling scheduling problems
by learning CP models from data. Previous works
that learn CP models 
assume that the data is tightly coupled 
with the CP formulation. 
For example, when learning from positive
examples only, the data at hand must contain
variables as they appear in CP models~\cite{beldiceanu2012model}. In other works,
it is assumed that
the data must include 
both positive and negative labels for
assignments that satisfy 
or violate constraints, respectively~\cite{lallouet2010learning}. 

In practice, for many real-world problems
existing data recordings do not 
relate to the CP formalism (e.g., does
not include variables). Instead, the data comes in the form of 
event logs,
which are logs that contain the activities that 
were executed, along with their 
start and completion times,
and 
information on resource consumptions~\cite[Chapter 4]{AalstBook}.
In this work, we address the challenge of learning 
scheduling models 
automatically from event logs without humans intervention.
 \begin{figure*}[t!]
	\centering
	\includegraphics[scale= 0.8]{Approach.pdf}
	\vspace{-0.5em}
	\caption{Our solution to learning scheduling models.}
	\label{fig:overview}
	%  \vspace{-0.8em}
\end{figure*}
Figure~\ref{fig:overview} presents an overview of our approach.
As our first step, we use existing techniques from process mining,
a rapidly developing
research field that aims at learning models
of the underlying process 
from event logs~\cite{AalstBook}. Specifically,
we apply a model learning method
that considers scheduled processes and their execution data~\cite{SenderovichRGMM15}. The result of the 
process mining step is a 
timed Petri net (TPN), a well-known formalism
for modeling dynamic schedule-driven systems~\cite{van1996petri}.
Since the TPN is highly expressive and 
captures synchronization, resource consumption, and temporal constraints, it cannot be used efficiently 
for scheduling the system that it models~\cite{van1996petri}. 
Therefore, in the next step we transform the learned 
TPN into a CP model. 
However, due to differences in the expressive power
of the two formalisms (TPN and CP), we must 
restrict the TPN to a family of models that can be 
transformed into CP formulations. To this end, we define a 
new type of TPNs, namely Seize-Delay-Release nets with resources (SDRR nets). We show a correct and complete algorithm that given a TPN verifies that it is an SDRR net. Subsequently, if the 
TPN is an SDRR net, we provide a 
correct and complete mapping of the SDRR net into a CP model,
thus completing our solution to learning CP models from event data. 
 


The main contribution of our work is threefold: \begin{itemize}
\item We provide an data-to-model solution that learns scheduling models from event logs.
\item We introduce a specialized type of timed Petri nets, namely
the SDRR nets, which enables a TPN to CP transformation. 
\item We show a correct and complete algorithm for verifying
that a TPN is an SDRR net
and provide a mapping of SDRR nets into CP models.

\end{itemize}

To demonstrate the usefulness of our approach,
we provide a two-part experimental evaluation.
In the first part, we learn 
models of job-shop scheduling
benchmarks using 
synthetically generated data. In the second part of the evaluation,
we learn scheduling models from real-world data
coming from a large outpatient cancel hospital in the United States. 
Given the data, our algorithms generate
the underlying appointment scheduling problem 
that we consequently solve 
using constraint programming.

 