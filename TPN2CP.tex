\label{TPN2CP}

In this part, we present a
transformation of schedule-driven Petri nets (SDPNs)
to constraint programming (CP) formulations. To this end, we define
the \emph{basic scheduling constructs set} (BSCS). Subsequently,
we show that the BSCS can be derived from the SDPN. Lastly, we
construct CP models from a given BSCS, thus completing the TPN2CP phase of our solution.  

\subsection{The Basic Scheduling Constructs Set}
The BSCS will contain the activities to be scheduled, the resources
along with their capacities, precedence constraints, and 
activity-and-resource dependent durations. In essence, the BSCS contains a set of building blocks of  
many well-known scheduling 
problems such as the 
job-shop scheduling problem (JSSP), 
and the resource-constrained project scheduling problem (RCPSP) 
(including its multi-mode variation). To define BSCS,
we follow the scheduling problem 
definition introduced by~\cite{van1996petri}.

\begin{mydef} [Basic Scheduling Constructs Set BSCS)] \label{def:sched}
Basic scheduling constructs are a set $B = \{ \mathcal{A}, \mathcal{R}, \varPi, c, d \}$ over a 
	over a finite time domain $\mathcal{T}$
	with,
	\begin{itemize}
		
		\item $\mathcal{A}$ being the set of activities to be scheduled,
		\item $\mathcal{R}$ being the set of renewable resources,
		\item $\varPi \subseteq \mathcal{A} \times \mathcal{A}$ being the precedence relation between pairs of activities,
		\item $c: \mathcal{R} \rightarrow  \mathbb{N}^{+}$ being the function that maps resources to their capacities, and, 
		\item $d: \mathcal{A} \times 2^{\mathcal{R}} \nrightarrow \mathcal{T}$ being the duration partial function that maps pairs of activities and resource sets (that can execute these activities) to values in the time domain.
		%	and, \item $b: \mathcal{A} \times 2^{\mathcal{R}} \times \mathcal{R} \nrightarrow \mathbb{N}^{+}$ being the demand partial function for resources when 
		%	executing an activity by a specific resource set. \todo{requires the addition of weights on the TPN!!}
	\end{itemize}
\end{mydef} \noindent A schedule $s$, which satisfies
a BSCS, is an allocation of resource sets to 
activities over time, i.e., 
$s \in \mathcal{A} \rightarrow  (2^\mathcal{R} \times \mathcal{T})$. A feasible schedule respects 
resource and precedence constraints. 
Without loss of generality, one often considers an objective function 
$\phi(s) \in \mathcal{R}^{+0}$ that 
assigns a real-valued number to a 
given schedule. In optimal scheduling one aims at finding a 
schedule that minimizes 
$\phi(s)$. In this work, we avoid learning
the objective function of the underlying scheduling problem, and hence assume that $\phi$ is given.

A prominent example that can be constructed using the BSCS is the \emph{job shop scheduling problem} (JSSP),
where only a single resource can perform each activity (hence $d(a) , a \in \mathcal{A}$ is sufficient to represent durations),
and resource capacities are equal to $1$. The objective function in
JSSP is to minimize the makespan.

\subsection{Deriving BSCS from SDPN}

In this part, we provide a 
mapping from Seize-Delay-Release 
timed Petri nets with resources (SDRR nets) into 
basic scheduling problems (BSPs). Specifically, 
given an SDRR net, $\mathcal{N} = \langle E, E',P, F, \tau \rangle$, our
we propose a mapping  that creates the corresponding BSP tuple, $\langle  \mathcal{A}, \mathcal{R},\varPi, c, d \rangle$.

For conciseness, 
we refer to elements of the TPN 
(e.g., places, transitions, flows)
when defining the BSP instead of labeling the TPN
and then using these labels to define the BSP. \todo{not sure that this needs to be said here}

For an SDRR net, we may reuse parts of
Algorithms~\ref{Alg1} and~\ref{Alg2} to derive the following
sets: (1) the set of SDCs $S = \{\mathcal{S}_1,\ldots,\mathcal{S}_m\}$,
(2) the set of resource places $P_r$, (3) 
the set of SDR nets that comprise the 
SDRR net $N_{sdr} = \{\mathcal{N}_i\}_{i=1}^{n}$ (4) the set of 
connector places $P_c = \{P_{c,i}\}_{i=1}^{n}$ for every SDR net
that comprises the SDRR net, and (5) the 
partition of SDCs, $\{C_i\}_{i=1}^{n}$, 
which includes sets of SDCs with common 
input and output connectors in the $i$th SDR net. 

\begin{mydef}[SDRR net to BSP Mapping]
	\begin{sloppypar} Given an SDRR net $(\mathcal{N} = (E, E', P, F, \tau), m_0)$ 
		and the sets $S, P_r,  N_{sdr}, P_{c,i}, C_i$,
		the BSP is constructed as follows:\end{sloppypar}
	\begin{itemize}
		\item The resource set is 
		given by the set of places, $\mathcal{R} = P_r$,
		\item The activity set corresponds to all connector places in $P_c$ except the sinks, namely $\mathcal{A} = \{p \in P_c | p\bullet \neq \emptyset \}$,
		\item The precedence
		relation $\varPi$ is constructed using
		the following: $$\varPi = \{(a,b) \in \mathcal{A} \times \mathcal{A} \ | \ a \rightsquigarrow b\},$$
		with $\rightsquigarrow$ indicating that there exists a path from $a$ to $b$,
		\item resource capacities, $c$, are equal to the
		initial marking
		of resource places, i.e., $\forall r \in \mathcal{R}: c(r) = m_0(p_r)$, and finally, 
		\item the duration (partial) function $d$ is computed as follows:
		\begin{align*} d = & \{ (a,R,d) \in \mathcal{A} \times 2^{\mathcal{R}} \times \mathcal{T} \ | \\
		& \forall \mathcal{S} \in S (\forall e \in E_{\mathcal{S}} \setminus E_{\mathcal{S}}'( a \in \bullet e  \ \land 
		\ R \subseteq (\bullet e \cap \mathcal{R})) \\ &\land \forall e' \in   E_{\mathcal{S}}' (d = \tau(e')) ) \} 
		\end{align*}	
	\end{itemize} 
\end{mydef}


\subsection{From BSCS to CP Models}
\todo{Kyle, please insert BSP to CP here. Chris will add a part on the intuition for scheduling people}

