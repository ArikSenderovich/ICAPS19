Solving scheduling problems
requires two main ingredients: a model 
that captures 
the essence of the underlying system and an algorithm that 
provides optimal solutions based on the model.
Constraint programming (CP) is a well-established
paradigm for modeling and solving deterministic
scheduling problems.  
It has been well-recognized that creating a suitable CP model is knowledge intensive 
task even when the underlying system is 
well-understood.  
In this work, we
aim at 
automating
the process of 
modeling scheduling problems
by learning CP models from event data.
To this end, 
we introduce
a novel methodology 
that combines process mining, 
timed Petri nets and CP.
As a first step, the approach
mines
timed Petri nets (TPNs) from event logs
that contain
executions of past schedules including
information on activities, timestamps and resources.
For the second step, 
we define a specialized TPN type,
namely the seize-delay-release net with 
resources (SDRR net),
which can be mapped into CP models. 
We provide a correct and complete algorithm
for detecting whether a TPN is an SDRR net.
Furthermore, we show a
mapping from SDRR nets into CP models.
Our approach provides an end-to-end solution
to model learning, going from 
event logs to model-based optimal schedules
without human intervention. 
To demonstrate the usefulness 
of the methodology we conduct
a series of experiments in which we
learn scheduling models from
two types of data: (1) event logs generated from 
job-shop scheduling benchmarks and (2) real-world event
logs that come from an outpatient hospital. 
