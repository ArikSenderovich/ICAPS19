
\documentclass[11pt]{article}

\usepackage{setspace}
\onehalfspacing
%\usepackage{mathtools}
\usepackage[ruled]{algorithm2e}
\newcommand{\todo}[1]{\textcolor{red}{\bf {#1}}}

\usepackage[a4paper]{geometry}
\setlength{\textwidth}{6in}
\usepackage[square,sort,comma,numbers]{natbib}
\usepackage[final]{pdfpages}
%\bibpunct{}{}{}{}{,}{;}
\pagestyle{plain}
%\usepackage{times}

\usepackage{graphicx,color}
\usepackage{amssymb, amsmath, amsthm, amsfonts} 
\usepackage{caption}
\usepackage{subfigure}
\usepackage{microtype,booktabs}
\usepackage{mathabx}
\usepackage{paralist,tabularx}
\usepackage[pagebackref=true]{hyperref}
%\usepackage{combine}
\hypersetup{colorlinks=true}
\usepackage{import}
\usepackage[ruled]{algorithm2e}
%\graphicspath{{./figures/}}
\graphicspath{{./figures/}{./figures/Paper1/}{.Paper1/figures/}}
\setcounter{tocdepth}{4}
\newtheorem{mydef}{Definition}
\newtheorem{myprop}{Proposition}
\usepackage{url}
\usepackage{balance}
%\usepackage{graphicx, color}
\newcommand\Item[1][]{%
	\ifx\relax#1\relax  \item \else \item[#1] \fi
	\abovedisplayskip=0pt\abovedisplayshortskip=0pt~\vspace*{-\baselineskip}}
\usepackage{booktabs}


%\newcommand{\RNum}[1]{\uppercase\expandafter{\romannumeral #1\relax}}
%\usepackage[ruled]{algorithm2e}
\graphicspath{{./Figures/}}
\newtheorem{prob}{Problem}
\newtheorem{mycor}{Corollary}
\newtheorem{mythm}{Theorem}
\newtheorem{myrem}{Remark}
\newtheorem{mylemm}{Lemma}
\newtheorem{myRun}{Running Example}
\DeclareMathOperator{\tot}{total}
\DeclareMathOperator{\argmax}{argmax}
\DeclareMathOperator{\argmin}{argmin}



	\title{\LARGE Data-Driven Scheduling Project: CP-PN}
	\author{}
	\date{\today}
	


\begin{document}
\maketitle

\section{Introduction}

Explanation as to why Petri nets are suitable representations of data, yet do not provide 
a suitable modeling language for the underlying scheduling problem(s).

Convey the main idea of Data into PN into CP.


%Petri nets are suitable to model resource dependencies and precedence constraints, which 
%are aspects that are difficult to elicit for scheduling modelers. 
%
%
%Typically, all inputs for a scheduling problem are given. However,
%often times the underlying system/process parameters
%are underspecified. With the right data, these parameters can be learned.
%Therefore, we transform transactional data into a PN that is then used as the "domain description" (domain as in Planning, not CP). 
%
%On top of the PN, we are given
%an underspecified scheduling problem (e.g., problem description 
%that 
%does not include precedence, resource 
%consumption, `real' processing times, etc.). We then
%exploit the learned Petri net to construct a fully specified problem
%and solve it using CP (deterministic approach for now).
%Solving the scheduling problem directly on the Petri net can be 
%an alternative -need to show experimentally that CP is better. 

%PN model systems and complex processes. These models can be discovered from the data. 
%One aims at scheduling resources in these complex processes. Resource-activity relations 
%are naturally captured by PN. We start with PN model, convert it to the corresponding Scheduling problem, and solve it.
%Alternative: use the PN directly. Explain why this is `worse'.



%\todo{can be extended to SPN for Queueing Network modeling (and scheduling) for stochastic problems. Interestingly, 
%	may require a different approach than what follows. May need to explain Petri net semantics.} 

\section{Motivating Examples}




\subsection{The Resource-Constrained Project Scheduling Problem}


\subsection{The Rooming Problem in Dana-Farber Cancer Institute}



\section{Preliminaries}

In this section we present 
the main two preliminaries for 
our work. Firstly, we define timed Petri nets, a special case of 
stochastic Petri nets as defined in ~\cite[Chapter 2]{haas2002stochastic},
having 
deterministic activity durations and routing. Secondly, we define a family of scheduling problems,
namely \emph{basic scheduling problems} (BSP).

\subsection{Timed Petri Nets}
Formally, a timed Petri net (TPN) is defined as follows. 

\begin{mydef} [Timed Petri net (TPN); TPN System] \label{def:tpn}
	A timed petri net $\mathcal{N}$ is a 
	tuple $\mathcal{N} = (E, E',P, F, \tau)$ with,
	\begin{itemize}
		\item $E$ being a finite set of transitions with $E' \subseteq E$ being a (possibly empty) set of timed transitions, 
		\item $P$ being a finite set of places,
		\item $F \subseteq E \times P \cup P \times E $ being the flow relation of the Petri net, and, 
		\item $\tau \in (E' \rightarrow \mathcal{T})$ being a function that maps deterministic durations
		to timed transitions;
	\end{itemize} A TPN system is a pair $(\mathcal{N}, m_0)$ with $\mathcal{N}$ being the structural part of the system, and $m_0: P \rightarrow \mathbb{N}$ being an initial marking of the net. 
\end{mydef} \noindent Note that the time domain $\mathcal{T}$ corresponds to 
Definition~\ref{def:sched}. We denote $\bullet p$ ($\bullet e$) the set of
transitions (places) that precede a place $p$ (a transition $e$), and by $p \bullet$ the 
set of transitions (places) that succeed a place $p$ (a transition $e$). 

The set of input (output) places is 
denoted $P_{input} \subseteq P$ ($P_{output} \subseteq P$), such that $\forall p \in P_{input} \ (\bullet p = \emptyset)$ ($\forall p \in P_{output} \ (p\bullet = \emptyset))$). We shall assume that $|P_{input}|>0$ and $|P_{output}|>0$ (TPN has at least a single input and a single output). Also, we assume that $|e\bullet|>1$ and $|\bullet e|>1$ (every transition is followed and preceded by at least one place). Finally, the structural part of the TPN ($\mathcal{N}$) can be represented by a graph, $\mathcal{G}(\mathcal{N}) = (P \cup T, F)$, with $P \cup T$ being its nodes, and $F$ being its edges.

How process mining learns TPNs from data (existing work on precedence mining, resource mining etc.)

\todo{Add explanation on Petri net semantics}


%
%\begin{proof}~\\
%\textbf{Case True}: The proof is straightforward. Line 11 verifies that every Petri net represented by $\{G_i\}_{i=1}^{m}$
%is an SD-net (by comparing every $G_i$ to Definition~\ref{def:sdnet}). This satisfies the first property
%of an SDR decomposition. Line 12 makes sure that $P_j$ and $P_r$ as returned by \emph{CandidateJobConnectorsResources()} satisfy properties 3,4 of an SDR decomposition, by checking the conditions on input and output sets in the original net ($\mathcal{N}$). Line 14 constructs a complete decomposition using SD constructs $\{G_i\}_{i=1}^{m}$ as $\mathcal{N}_1, \ldots, \mathcal{N}_{n-2}$ (with $n = m+2$ since we add two additional nets for job connectors and resource places). It also creates $\mathcal{N}_r$ and $\mathcal{N}_j$ with only place sets (no transitions), thus satisfying property 2 of SDR decomposition. The decomposition is complete with respect to $F_D = F_j\cup F_r$ due to equations 3-5 (no additional flows introduced). Since the decomposition is complete and SDR, the TPN is an SDR-Net.
%
%\todo{Add algorithm for the missing functions}
%
%
%
%\textbf{Case False:}
%We prove by contradiction. Assume for the sake of contradiction that Algorithm~\ref{Alg1} returns False, 
%yet $\mathcal{N}$ is an SDR-TPN. Since the TPN is SDR-TPN, there exists a decomposition $D_{sdr}'((\mathcal{N}, m_0)) = \{(\mathcal{N}_1', m_0^1), \ldots, (\mathcal{N}_n', m_0^{n-2}) , (\mathcal{N}_r', m_0^{r}), (\mathcal{N}_j', m_0^{j})$ such that the conditions (1-4) in Definition~\ref{def:sdr-tpn} hold. We denote $P_j'$ and $P_r'$ the job connector places
%and the resources places of $D_{sdr}'$. 
%
%If the algorithm returned False, then one of the following 
%conditions must hold:
%\begin{itemize}
%	\item[C1.] One of the Petri nets represented by $\{G_i\}_{i=1}^m$ that was returned by \emph{ConnectedComponenets()} is not an SD net, or, 
%	\item[C2.] One of the sets $P_j, P_r$ returned by \emph{CandidateJobConnectorsResources()} is not a job connector set or a resource set, respectively.
%\end{itemize}


%
%First, we prove that C1 leads to a contradiction. We consider the sets $P_j$ and $P_r$ as they are returned by $\emph{CandidateJobConnectorsResources()}$.
%All places of $\mathcal{N}$ that have a single immediate transition as their input transitions, and a single timed transition as their output transitions 
%will not be contained in $P_j$ and $P_r$, yet these places will reside in $P_{sd}$ (line 6 of Algorithm~\ref{Alg1}). Hence, every place in $P_{sd}$, 
%will have a 


%
%
%Let $G_i \in \{G_i\}_{i=1}^m$ be the graph
%representation of a Petri net that violates the SD condition. Since the undirected version of $G_i$ is a connected component, then
%$G$ represents a fully connected Petri net. To violate the SD condition, one of the following SD violations must hold:
%\begin{itemize}
%	\item [V1] There is one place in $P_i$, yet there is at least a single redundant transition $e'' \in E_i$ in the set of transitions of $G_i$. 
%	since $G_i$ is connected, 
%	there is at least one $(x,y)\in F_i \ | x = e'' \lor y = e''$. If $|P_i|=1$ 
%	(single place), then $e''$ is connected to $p \in P_i$. 
%\end{itemize} 


%
%
%We know that the alg returned false if the connected component is not sd, or it is sd but the places pj and pr do not correspond to the original places.
%
%
%
%
%\end{proof}


\subsection{Basic Scheduling Problems}
 
In this part, we define
a family of scheduling problems that we
refer to as basic scheduling problems (BSPs). (Note: we consider only deterministic
scheduling problems, at this point.)

The BSP is adapted in its expressiveness to 
to Timed Petri nets, which we use in order
to model the data. The BSP covers a broad spectrum of scheduling problems as it 
generalizes the 
job-shop scheduling problem (JSP), 
and the resource-constrained project scheduling problem (RCPSP) (including its multi-mode variation).

\begin{mydef} [Basic Scheduling Problem (BSP)] \label{def:sched}
A basic scheduling problem is a tuple $\langle  \mathcal{A}, \mathcal{R}, \varPi, c, \rho, d, b \rangle$ over a 
over a finite time 
with,
\begin{itemize}
 
	\item $\mathcal{A}$ being the set of activities,
	\item $\mathcal{R}$ being the set of renewable resources,
	\item $\varPi \subseteq \mathcal{A} \times \mathcal{A}$ being the precedence relation between activities,
	\item $c: \mathcal{R} \rightarrow  \mathbb{N}^{+}, r \in \mathcal{R}$ being the function that maps resources to their capacities, 
	\item $\rho: \mathcal{A} \rightarrow 2^{\mathcal{R}}$ being the resource to activity assignment (RTAA)  function that maps activities
	to sets of resources, 
	\item $d: \mathcal{A} \times 2^{\mathcal{R}} \nrightarrow \mathcal{T}$ being the duration partial function that maps pairs of activities and resource sets to values in the time domain, 
	and, \item $b: \mathcal{A} \times 2^{\mathcal{R}} \times \mathcal{R} \nrightarrow \mathbb{N}^{+}$ being the demand partial function for resources when 
		executing an activity by a specific resource set. \todo{requires the addition of weights on the TPN!!}
\end{itemize}
\end{mydef} \noindent A schedule $s$, which is a solution to 
the BSP, comprises an allocation of resource sets to 
activities over time, i.e., 
$s \in \mathcal{A} \rightarrow  (2^\mathcal{R} \times \mathcal{T})$. A feasible schedule respects resource and precedence constraints. Without loss of generality, one often considers a score function 
$\phi(s) \in \mathcal{R}^{+0}$ that 
assigns a real-valued number to a 
given schedule. We would often seek a 
schedule that minimizes 
$\phi(s)$.

It is trivial to see that the 
\emph{Resource-Constrained Project Scheduling Problem} 
(RCPSP) with renewable resources \cite{BRUCKER1993} is a 
special case of the BSP. By setting $\rho(a)$ to return singleton subsets (one
set of resources is chosen to perform an activity)
of renewable resource sets $\mathcal{R}$, and taking
$\phi(\cdot)$ to be the makespan, we arrive at the definition of 
RCPSP as it is presented in~\cite{BRUCKER1993}. 

Furthermore, without constraining $\rho(a)$ to return singletons, 
every element in $\rho(a)$ can be mapped to a unique 
mode per activity, which leads to a more general problem, namely
the multi-model RCPSP (MMRCPSP).



\section{Timed Petri Net Decompositions}

%In what follows, we shall assume that the graph is connected, in the sense that there is a path from every input place to at least one output place. \todo{this can be formalized by defining a closure and assuming that the graph is strongly connected.} 
%

In this section, we present the notion of decompositions for Petri nets.
Specifically, we consider decompositions that will enable us to 
map a timed Petri net into a basic scheduling problem.

We start with the notion of complete TPN decomposition. 
Then, we introduce a special case of the complete decomposition, 
namely the \emph{seize-delay-release} (SDR) decomposition. Figure~\ref{def:sdr-tpn}
will serve as a running example for this part.

\begin{figure}[t!]
	\centering
	\includegraphics[scale=0.8]{SDR_TPN_ex1_crop.pdf}
	\caption{Seize-Delay-Release Net (Definition~\ref{def:sdr-tpn})}
	\label{fig:sdr-net}  
	%  \vspace{-1em}
\end{figure}
\begin{mydef} [Complete Decomposition of TPN] \label{def:decompose_n}
	\begin{sloppypar}A complete decomposition of a TPN $(\mathcal{N}, m_0)$ with respect to a 
		set of flows $F_D \subseteq F$ is a finite set $D((\mathcal{N}, m_0)) = \{(\mathcal{N}_1, m_0^1), \ldots, (\mathcal{N}_n, m_0^n)\}$ with $\mathcal{N}_i = (E_i, E'_i,P_i, F_i, \tau_i, i \in I$ ($I$ being an index set) and 
		a set of flows $F_D$ such that \end{sloppypar} \begin{itemize}
		\item $\bigcap_{i=1}^n E_i = \emptyset,\bigcap_{i=1}^n P_i = \emptyset$ (no shared places and transitions between the components),
		\item $\bigcup_{i=1}^n E_i = E, \bigcup_{i=1}^nP_i = P$, (decomposition is lossless in places and transitions),
		\item $\forall i = 1,\ldots, n: F_i \subseteq F$, (no new flow pairs within components),
		\item $\forall i = 1,\ldots, n: \tau_i(e) = \tau(e), e \in E_i$, (time preservation), 
		\item $\forall i = 1,\ldots, n:  m_0^i(p) = m_0(p), p \in P_i$, (marking preservation), and, 
		\item there exists a set of flows $F_D \subseteq F: \bigcup_{i=1}^n F_i \cup F_D = F$. 
	\end{itemize}
\end{mydef} \noindent The set $F_D$ is called the complementary flow set. 
Trivially, for any given TPN a set that contains the TPN itself is a complete decomposition 
with $F_D = \emptyset$. Furthermore, given a place $p$ we define $$I_{\bullet p} = \{i \in I \ | \ \exists e \in \bullet p(e \in E_i)  \},$$ and $$I_{p \bullet}  = \{i \in I \ | \ \exists e \in p\bullet(e \in E_i)  \}, $$
as index sets of input and output SD nets of place $p$. 

\subsection{Seize-Delay-Release Timed Petri Net}




\begin{mydef} [Seize-Delay net] \label{def:sdnet}
	An Seize-Delay net (SDN) is a timed Petri net, 
	$\mathcal{N}_{SD}= (E, E' , P , F, \tau)$, such that 
	\begin{itemize}
		\item The set $E = \{ e_{seize}, e_{delay} \}$ contains seize and delay transitions, 
		\item The set $E' = \{e_{delay}\}$ contains a single delay transition, 
		\item The set $P = \{p_{delay}\}$ is a single delay place, and,
		\item The flow $F = \{ (e_{seize}, p_{delay}), (p_{delay}, e_{delay})\}$ is as presented in Figure~\ref{fig:sd-net}.
	\end{itemize}
\end{mydef} 
\todo{The timed transition itself can be a complex PN construct (hierarchy), as long as it does not
	change resource consumption or the SD-ness of the skeleton.}

\begin{figure}[t!]
	\centering
	\includegraphics[scale=0.3]{sd-net.png}
	\caption{Seize-Delay Net (Definition~\ref{def:sdnet})}
	\label{fig:sd-net}  
	%  \vspace{-1em}
\end{figure}



\todo{Should we add that SDR TPN produces a DAG? loops make it difficult to assume precedence order}
\begin{mydef}  [Seize-Delay-Release timed Petri net (SDR-TPN)]  \label{def:sdr-tpn}
	\begin{sloppypar} A TPN $(\mathcal{N} = (E, E',P, F, \tau), m_0)$ is an SDR-TPN if there exists a complete decomposition (that we refer
		to as SDR decomposition) $D_{sdr}((\mathcal{N}, m_0)) = \{(\mathcal{N}_1, m_0^1), \ldots, (\mathcal{N}_n, m_0^{n-2}) , (\mathcal{N}_r, m_0^{r}), (\mathcal{N}_j, m_0^{j})\}$ with the following properties: \end{sloppypar}
	
	\begin{enumerate} 
		\item The components $\mathcal{N}_i, i \in I^n = \{1,\ldots, n-2\}$ are seize-delay nets with $I^n$ being their index set,
		\item $\mathcal{N}_r$ and $\mathcal{N}_j$ are TPNs that contain only places with place sets $P_r$ and $P_a$, respectively, ($E_r = E_j = \emptyset, F_r = F_j = \emptyset$), 
		
		
		\item $P_a = \{ \ p \in P \ | \ \bullet p \subseteq E' \ \land \ p\bullet \subseteq E \setminus E' \land I_{\bullet p}^n \cap I_{p \bullet}^n = \emptyset \},$
		%\item  $P_j = \{ \ p \in P \ | \ \exists L\neq\emptyset, K\neq\emptyset \subset I \ (p\bullet \subseteq E_l \setminus E'_l, \forall l \in L \ \land \ \bullet p \subseteq E'_k, \forall k \in K \ \land \ L \cap K = \emptyset )\}$, and,
		
		\item $P_r = \{ \ p \in P \ |  \bullet p \subseteq E' \ \land \ p\bullet \subseteq E \setminus E' \ \land \ I_{\bullet p}^n = I_{p \bullet}^n  \land \ |\bullet p| = |p \bullet| \geq 1\}$.
		
		\item $\forall e \in E \setminus E'(|\{p \in P \ | \ p \in \bullet e \land p \in P_a   \}| =1)$.
		
	\end{enumerate}
	
	\noindent \todo{change to activity connectors?} The places in $P_a$ are referred to as activity connectors, while  
	the places in $P_r$ are referred to as the resources. 
	\label{SDRTPN}
\end{mydef} \noindent Below, we explain Properties 3 - 5
of the SDR decomposition. Property 3 requires 
that
every place in $P_a$ must connect between a subset
of the SDNS and another (different) subset of
SDNs. Note that TPN input places ($P_{input}$)
satisfy the condition 
if they are connected to a subset of SD nets.
Similarly, TPN output places, $P_{output}$, will reside 
in $P_a$ if their predecessors
come from a subset of the SD nets of $D_{sdr}$. 
Property 4 is similar to Property 3 in that
places in $P_r$ connect between SD nets. However,
for a place to be a resource it needs
to connect the timed transitions of a set of SD nets
to the immediate transition of the same set of SD nets. Clearly, 
more than a single SDN may be connected to both 
the activity connectors and the resources.
Property 5 requires that every SD net of the decomposition 
has exactly a single precondition \todo{explain why}.

Furthermore, it is easy to see that the decomposition is complete with 
respect to a complementary flow set, 
$$F_D = \{ (x,y) \in F \ | 
\ x\in P_a \cup P_r \lor y \in P_a \cup P_r \}.$$


Figure~\ref{fig:sdr-net} provides an example for an SDR-TPN. Specifically, 
it revisits the DFCI example, showing two patients who are waiting for an examination activity.
The patients may require two subsets of resources...

%Now we prove SDR-TPN -> Connected components are SD. 
\noindent In the next part, we propose Algorithm~\ref{Alg1} that  
takes a TPN as its input, and returns
whether the TPN is SDR-NET; if the output is
positive, 
the algorithm returns
a complete decomposition that 
corresponds to Definition~\ref{SDRTPN}. 

\noindent \todo{Go over the Algorithm line by line}


\noindent To prove the correctness of Algorithm~\ref{Alg1},
we shall first prove the following Lemma.

\begin{mylemm} \label{lem1}
	\begin{sloppypar}
		Let $(\mathcal{N}, m_0)$ be an SDR-TPN with an SDR
		decomposition $D_{sdr}((\mathcal{N}, m_0)) = \{(\mathcal{N}_1, m_0^1), \ldots, (\mathcal{N}_n, m_0^{n-2}) , (\mathcal{N}_r, m_0^{r}), (\mathcal{N}_j, m_0^{j})\}$ with an activity connector place set $P_a$ and resource set $P_r$. 
		Denote $P_{sd} = \bigcup_{i\in I^n} P_i$ the union of place sets that correspond to the ($n-2$) SD nets $\mathcal{N}_i, i \in I^n$. Then, the two statements hold: \begin{enumerate}
			\item The set $P_{sd}'$ as defined in line 2 of Algorithm~\ref{Alg2} is equal to $P_{sd}$. 
			\item The sets $P_a'$ and $P_r'$ as returned by Algorithm~\ref{Alg2} are equal to $P_a$ and $P_r$, respectively.  
		\end{enumerate}
	\end{sloppypar}
\end{mylemm}
\begin{proof}
	
	First, we prove Lemma~\ref{lem1}, part 1, by means of bidirectional inclusion. 	
	
	\noindent Assuming that $p \in P_{sd}$, there exists
	$i \in I^n$ (an index of an SD net) such that 
	$|\bullet p| = |p \bullet| = 1$ (by Definition~\ref{def:sdnet}), 
	and $$\bullet p \subseteq E_i \setminus E'_i \ \land \ p\bullet \subseteq E'_i.$$
	Hence, $\bullet p \subseteq E \setminus E' \ \land \ p\bullet \subseteq E'$, which 
	implies that $p \in P_{sd}'$ by definition of $P_{sd}'$ in line 2 of Algorithm~\ref{Alg2}. 
	
	\noindent Assuming that $p \in P_{sd}'$, 
	$p$ satisfies the following condition (from line 2 of Algorithm~\ref{Alg1}): \begin{equation} 
	\label{eq:sd}|\bullet p| = |p \bullet| = 1 \ \land \ \bullet p \subseteq E \setminus E' \ \land \ p\bullet \subseteq E'.\end{equation} Here, we show that every place $p \in P_a \cup P_r$ necessarily does 
	not satisfy Eq.(\ref{eq:sd}), 
	and hence it must be that $p \in P_i, i \in I^n$, which leads to $p \in P_{sd}$. Specifically, all non-input and non-output 
	places in $P_a$ satisfy $\bullet p \subseteq E'$ and hence Eq.(\ref{eq:sd}) does not hold. The input and output places
	in $P_a$ do not satisfy $|\bullet p| = |p \bullet| = 1$ (which again contradicts Eq.(\ref{eq:sd})). 
	Furthermore, every place in $P_r$ satisfies $\bullet p \subseteq E'$ which also 
	contradicts Eq.(\ref{eq:sd}). Hence, if $p \in P_{sd}'$ then $p \in P_i$ with 
	$P_i$ being a place set of $\mathcal{N}_i$ (an SD net of $D_{sdr}$). This proves that
	$p \in P_{sd}$. 
	
	From the proof of Lemma~\ref{lem1}, part1, we get that $P_{cand}$ from line 3 of Algorithm~\ref{Alg2} is
	equal to $P_a \cup P_r$. This follows from the fact that places 
	of SDR-TPN can be either in $P_{sd}$ or in $P_a \cup P_r$. Since $P_{sd}' = P_{sd}$ and $P_{cand} = P \setminus P_{sd}'$,
	we arrive at $P_{cand} = P_a \cup P_r$. In what follows, we exploit this result to prove the second part of Lemma~\ref{lem1}, namely that $P_a' = P_a$ and $P_r' = P_r$. The proof is based on the following two claims: \begin{enumerate}
		\item Trivially, any $p \in P_{cand}$ satisfies $p \in P_a \ \lor p \in P_r$,
		since $p \notin P_i$ for $\mathcal{N}_i, i \in I$.
		\item $P_r' \cap P_a = \emptyset$. We prove
		this by showing that if $p \in P_a$ then $p \notin P_r'$, and vice versa. 
		
		If $p \in P_a$, then by definition we know
		that $I_{\bullet p}^n \cap I_{p \bullet}^n = \emptyset$. This implies that 
		transitions in $\bullet p$ and $p \bullet$ belong 
		to two non-intersecting sets of SD constructs say $\{\mathcal{N}_k\}_{k \in I_{\bullet p}^n}$ and 
		$\{\mathcal{N}_l\}_{l \in I_{p\bullet}^n}$, respectively. However, if $p \in P_r'$ then
		we would get 
		$\forall e \in \bullet p 
		(prec(e) \cap p \bullet = \emptyset)$ (definition of $P_r'$ in line 5 of Algorithm~\ref{Alg2}),
		which would imply that $\exists i \in I^n: i \in I_{\bullet p}^n \land i \in I_{p\bullet}^n$ which would contradict 
		that $I_{\bullet p}^n \cap I_{p \bullet}^n = \emptyset$. Therefore, we get that if $p \in P_a$ then $p \notin P_r'$.
		
		If $p \in P_r'$,
		by definition in line of Algorithm, $\forall e \in \bullet p 
		(prec(e) \cap p \bullet = \emptyset)$. This implies that there is at least one SD net, 
		say $\mathcal{N}_i$, such that $i \in I_{\bullet p}^n \land i \in I_{p\bullet}^n$,
		which implies that $p \notin P_a$.
	\end{enumerate} From these two claims we get that $P_a' = P_a$ and $P_r' = P_r$, since $P_r \cup P_a = P_r' \cup P_a'$.
\end{proof}

\noindent The correctness of Algorithm~\ref{Alg1} follows from the combination of Lemma~\ref{lem1}, and the following  proposition. 
\begin{mythm} \label{prop:dsdr}
	Let $(\mathcal{N} = (E, E', P, F, \tau), m_0)$ be a TPN. 
	The decomposition $D_{sdr}'$ returned by Algorithm~\ref{Alg1} 
	is an SDR decomposition
	if and only if $(\mathcal{N} = (E, E', P, F, \tau), m_0)$ is an SDR-TPN.
\end{mythm}

\begin{proof} The first direction, $D_{sdr}'$ is SDR Decomposition $\Rightarrow$ SDR-TPN, follows immediately from Definition~\ref{SDRTPN}. Assume that the 
	complete decomposition returned by Algorithm~\ref{Alg1}, namely $D_{sdr}'$, 
	is an SDR decomposition. Since $(\mathcal{N},m_0)$ is a 
	TPN and we found an SDR decomposition, 
	namely $D_{sdr}'$, the TPN is SDR-TPN, according to Definition~\ref{def:sdr-tpn}. \\
	
	
	\begin{sloppypar} We now prove the second direction. 
		Assume that $(\mathcal{N} = (E, E', P, F, \tau), m_0)$ is an SDR-TPN.
		We need to prove that $D_{sdr}'$ (returned by Algorithm~\ref{Alg1}) is 
		an SDR decomposition of the TPN. Since $\mathcal{N}$ is an SDR-TPN, 
		it has an SDR decomposition denoted 
		$D_{sdr}((\mathcal{N}, m_0)) = \{(\mathcal{N}_1, m_0^1), \ldots, (\mathcal{N}_n, m_0^{n-2}) , (\mathcal{N}_r, m_0^{r}), (\mathcal{N}_j, m_0^{j})\}$ with the corresponding notation and 
		conditions applying from 
		Definition~\ref{SDRTPN}.  \end{sloppypar}
	
	From Lemma~\ref{lem1} we know that since $\mathcal{N}$ is an SDR-TPN,
	the set $P_{sd}'$ computed in line 2 of Algorithm~\ref{Alg2}
	is equal to $P_{sd}$ and that  
	$P_a'$ and $P_r'$ as computed in lines 5-6 of 
	Algorithm~\ref{Alg2}
	are equal to $P_a$ and $P_r$ of $D_{sdr}$
	(the place sets 
	of $\mathcal{N}_j$, and $\mathcal{N}_r$), 
	respectively.
	
	We are remained to prove that running 
	connected components on the undirected representation of
	$\mathcal{G}(\mathcal{N}_{sd})$ results in ($n-2$)
	SD nets. This is straightforward, as 
	$\mathcal{N}_{sd}$ (line 8 in Algorithm~\ref{Alg1}) 
	contains all transitions and places of 
	the SDR-TPN, except $P_a \cup P_r$, we know that it comprises ($n-2$) SD nets that 
	are not connected among themselves.
	
	By definition of connected components,
	we get that the ($n-2$) SD nets will not overlap 
	in terms of places, transitions, and flows.
	Furthermore, the union of places, transitions and flows, 
	of the ($n-2$) components,
	with $P_a', P_r', F_j', and F_r'$ will result $\mathcal{N}$. Thus, this will satisfy 
	the first two properties in Definition~\ref{def:sdr-tpn}. Properties 3 and 4
	are satisfied due to Lemma~\ref{lem1}, and hence we get that if 
	$\mathcal{N}$ is an SDR-TPN, $D_{sdr}'$ is an SDR decomposition.
\end{proof}




\begin{mycor}
	Given a TPN $(\mathcal{N},m_0)$, Algorithm~\ref{Alg1} is correct, i.e., if it returns $\emptyset$ then the TPN 
	is not an SDR-TPN, and if it returns $D_{sdr}'$, then the TPN is an SDR-TPN, and $D_{sdr}'$ is an SDR decomposition.
\end{mycor}
\begin{proof}
	If the algorithm returns false, $D_{sdr}'$ is not an SDR decomposition. From Proposition~\ref{prop:dsdr},
	we get that  $(\mathcal{N},m_0)$ is not an SDR-TPN. If the algorithm returns $D_{sdr}'$, since it is an SDR decomposition
	by construction (verified by the Algorithm), then Proposition~\ref{prop:dsdr} implies that
	$(\mathcal{N},m_0)$ is an SDR-TPN.
\end{proof}
\todo{intuition: can prove uniqueness of sdr decomposition, up to the labeling of the sets}

\section{Mapping SDR-TPN to BSP} 



In this part, we provide a 
mapping from Seize-Delay-Release 
timed Petri nets (SDR-TPNs) into 
basic scheduling problems (BSPs). Specifically, 
given an SDR-TPN, we 
need to create a tuple $\langle  \mathcal{A}, \mathcal{R}, \varPi, c, \rho, d, b \rangle$ that 
is the formal representation of the corresponding BSP.

For conciseness, 
we refer to elements of the TPN 
(e.g., places, transitions, flows)
when defining the BSP instead of labeling the TPN
and then using these labels to define the BSP. \todo{not sure that this needs to be said here}



\begin{mydef}[SDR-TPN to BSP Mapping]
\begin{sloppypar} Given an SDR-TPN $(\mathcal{N} = (E, E', P, F, \tau), m_0)$ 
and its SDR decomposition 
$D_{sdr}((\mathcal{N}, m_0)) = \{(\mathcal{N}_1, m_0^1), \ldots, (\mathcal{N}_n, m_0^{n-2}) , (\mathcal{N}_r, m_0^{r}), (\mathcal{N}_j, m_0^{j})\}$ with $P_a$ being the activity connector places, and $P_r$ the resource places,
the BSP is created as follows:\end{sloppypar}
\begin{itemize}
	\item The resource set is equal 
	to the resource places, $\mathcal{R} = P_r$,
	\item Activities $\mathcal{A}$ and their precedence
	relation $\varPi$ are constructed using Algorithm~\ref{ConstructActPrec},
	\item resource capacities are equal to the 
	initial marking
	of resource places, $\forall r \in \mathcal{R}: c(r) = m_0(p_r)$, and finally, 
	\item the RTAA and the duration functions $\rho, d, b$ are computed using Algorithm~\ref{ConstructRhoDur}.	
\end{itemize} 
\end{mydef}



Define SDR-TPN without resource places. Then aggregation over SDR-TPN without resource places, 
according to xor splits into an activity diagram. Then, one can use connected components and then strongly
connected components to elicit precedence. 




The idea is that every pair $a, R: R \in \rho(a)$ 
should correspond to one transition (SD construct). 












%\begin{mydef} [Conflicting Place; Conflict-Free TPN; Time Conflict-Free TPN] \label{def:conf_p}
%	Given a TPN $\mathcal{N}$, a place $p \in P$ is said to be conflicting if $|O_t(p)|>1$. 
%	The subset of conflicting places in a TPN is denoted $P_c \subseteq P$. A TPN 
%	for which $P_c = \emptyset$ is 
%	said to be conflict-free. If $\forall p \in P_c, O_t(p) \subseteq 
%	E\setminus E'$ then 
%	the TPN is a Time Conflict-Free TPN. 
%\end{mydef} 

%\begin{mydef} [Branching Transitions; Branch-Free TPN; Time Branching TPN] \label{def:branch_e}
%	Given a TPN $\mathcal{N}$, a transition $e \in E$ is said to be branching if $|O_p(e)|>1$. 
%	The subset of branching transitions in a TPN is denoted $E_b \subseteq E$. A TPN 
%	for which $E_b = \emptyset$ is 
%	said to be branch-free. If $\forall e \in E_b: e \in E'$ then 
%	the TPN is said to be a Time Branching TPN. 
%\end{mydef} 
%
%\begin{mydef} [Joining Transitions; Join-Free TPN; Immediate Joining TPN] \label{def:join_e}
%	Given a TPN $\mathcal{N}$, a transition $e \in E$ is said to be joining if $|I_p(e)|>1$. 
%	The subset of branching transitions in a TPN is denoted $E_j \subseteq E$. A TPN 
%	for which $E_j = \emptyset$ is 
%	said to be join-free. If $\forall e \in E_j: e \in E \setminus E'$ then 
%	the TPN is said to be an Immediate Joining TPN. 
%\end{mydef} 
%
%\todo{Not sure if we need the state machine definition here - but maybe for more complex problems}
%
%\begin{mydef} [State Machine]
%	A TPN $\mathcal{N}$ is a state-machine if it is branch-free and join-free.	
%\end{mydef} State machines allow for choices (conflicts), 
%but do not allow for concurrency. Verifying that a Petri net is a state machine is polynomial in the number of transitions.  

%\noindent Next, we turn to characterize transition types and constructs in TPN.
%\begin{mydef} [Seize; Delay; Delay-Release; D-SDR Net Net] \label{def:types_e}
%	Given a Time Branching and an Immediate Joining TPN $\mathcal{N}$, a transition $e \in E$ is said to be: \begin{itemize}
%	\item a seize transition $\iff e \in E \setminus E'$ (an immediate transition),
%	\item a delay transition $\iff e \in E' \setminus E_b$ (timed; no branching),
%	\item a delay-release transition $\iff e \in E_b$ (timed; branching).
%\end{itemize} 
%\end{mydef} 
%








\bibliographystyle{plainnat}
\bibliography{Biblio}
\newpage

\section{Appendix A: Algorithms}



\begin{algorithm}[h!]
	
	\LinesNumbered
	\DontPrintSemicolon
	\SetAlgoLined
	\KwIn{TPN System $(\mathcal{N} = (E, E',P, F, \tau), m_0)$}
	\KwOut{Decomposition $D_{sdr}((\mathcal{N}, m_0))$ or $\emptyset$}
	\Begin{
		
		\tcc{Step 1: Find the candidates to be activity connectors and resources and detatch them from the net. The function CandidateActivityConnectorsResources()
			takes a TPN and returns two sets (activity and resource sets) that 
			have the potential to satisfy conditions (3a) and (3b). (see Algorithm~\ref{Alg2})}
		
		$P_a', P_r' \leftarrow$ \emph{CandidateActivityConnectorsResources}($\mathcal{N}$);\\	
		$F_j' = \{(x,y) \in F | x\in P_a' \lor y \in P_a'  \}$;\\
		$F_r' = \{(x,y) \in F | x\in P_r' \lor y \in P_r'  \}$;\\
		
		
		$F_{sd}  = F \setminus (F_j' \cup F_r')$;\\
		$P_{sd} = P\setminus (P_a' \cup P_r')$;\\
		$E_{sd} = E$; $E'_{sd} = E'$; $\forall e \in \tau_{sd}(e) = \tau(e)$;\\
		$\mathcal{N}_{sd} = (E_{sd}, E'_{sd}, P_{sd}, F_{sd}, \tau_{sd})$;\\
		
		\tcc{Step 2: The function ConnectedComponents takes the 
			graph representation of TPN, $\mathcal{G}(\mathcal{N}_{sd})$, and returns $m$
			Petri nets that correspond to the connected components of the graph.}
		$G_{sd} = \mathcal{G}(\mathcal{N}_{sd})$;\\
		$\{\mathcal{N}_i'\}_{i=1}^{m}$ = \emph{ConnectedComponents}(\emph{Undirected}($G_{sd}$);\\
		
		\tcc{Step 3: The function ConstructDecomposition() takes the connected components, along 
			with the initial marking and the sets $P_a'$ and $P_r'$ and constructs
			a complete decomposition of size $m+2$.}
		
		$D_{sdr}' = $\emph{ConstructDecomposition}($\{\mathcal{N}_i'\}_{i=1}^{m}, m_0, P_a', P_r'$) \;
		
		\tcc{Step 4: The boolean function VerifySDEDecomposition() takes a complete TPN decomposition and 
			checks whether it is SDR based on the four properties of Definition~\ref{def:sdr-tpn}.}
		
		\uIf{\emph{VerifySDRDecomposition}($D_{sdr}'$) = True}
				{\Return $D_{sdr}'$ \; }
				\Else{ \Return $\emptyset$\;}
			
	}
	\caption{Checks if TPN is an SDR-TPN and return a complete decomposition.}
	\label{Alg1}
\end{algorithm}

\begin{algorithm}[t!]
	
	\LinesNumbered
	\DontPrintSemicolon
	\SetAlgoLined
	\KwIn{TPN $\mathcal{N} = (E, E',P, F, \tau)$}
	\KwOut{Two sets: $P_a'$ and $P_r'$.}
	\Begin{
		
		\tcc{Detect all seize-delay places. Deduce candidates to be activity connectors or resource places.}
		$P_{sd} = \{p \in P \ | \ |\bullet p| = |p \bullet| = 1 \ \land \ \bullet p \subseteq E \setminus E' \ \land \ p\bullet \subseteq E'\}$;\\
		$P_{cand} = P \setminus P_{sd}$;\\
		
		$\forall e: prec(e) = \{ e' \in E \ | \ \exists p\in P ((e',p) \in F) \ \land \ (p,e) \in F \}$;\\	
		
		\tcc{Divide the candidates into two sets: resource places that connect transitions from... and activity connectors.}
		
		$P_r' = \{ p \in P_{cand} \ | \ \bullet p \neq \emptyset \ \land \ p\bullet \neq \emptyset \ \land \ \forall e \in \bullet p ( \ prec(e) \subseteq p \bullet \ )\}$;\\
		$P_a' = P_{cand} \setminus P_r'$;\\
		\Return $P_a', P_r'$;\\
	}
	\caption{Returns candidate place sets for activity connectors and resources.}
	\label{Alg2}
\end{algorithm}




\begin{algorithm}[t!]
	
	\LinesNumbered
	\DontPrintSemicolon
	\SetAlgoLined
	\KwIn{SDR-TPN $(\mathcal{N} = (E, E', P, F, \tau), m_0)$, SDR decomposition 
		$D_{sdr}((\mathcal{N}, m_0)) = \{(\mathcal{N}_1, m_0^1), \ldots, (\mathcal{N}_n, m_0^{n-2}) , (\mathcal{N}_r, m_0^{r}), (\mathcal{N}_j, m_0^{j})\}$, $P_a$, $P_r$.}
	\KwOut{Set of activities and precedence order, $\mathcal{A}, \varPi$.}
	\Begin{
		
		\tcc{Find activities from sets of SD constructs with
			joint input place. Every input place 
			into an SD construct corresponds to one activity.}
		$\mathcal{A} = \{p \in P \ | \ \exists \ e \in E \setminus E'(p \in \bullet e) \}$;\\
		\tcc{Create a new without resource places and flows between resources places and 
			SD constructs. The function $Follows(G, a, a')$ returns True if
			there is a path from $a$ to $a'$ in a directed graph $G$.}
		$F_r = \{(x,y) \in F | x\in P_r \lor y \in P_r  \}$;\\
		$\mathcal{N}'= (E, E', P \setminus P_r, F \setminus F_r, \tau), m_0)$;\\
		$\varPi = \{(a,a') \in \mathcal{A} \times \mathcal{A} \ | \ Follows(\mathcal{G}(\mathcal{N}'), a, a')  =True$ \};\\
		\Return $\mathcal{A}, \varPi$;\\
	
	}
	\caption{Constructs activity set $\mathcal{A}$ and precedence relation $\varPi$.}
	\label{ConstructActPrec}
\end{algorithm}






\begin{algorithm}[t!]
	
	\LinesNumbered
	\DontPrintSemicolon
	\SetAlgoLined
	\KwIn{SDR-TPN $(\mathcal{N} = (E, E', P, F, \tau), m_0)$, SDR decomposition 
		$D_{sdr}((\mathcal{N}, m_0)) = \{(\mathcal{N}_1, m_0^1), \ldots, (\mathcal{N}_n, m_0^{n-2}) , (\mathcal{N}_r, m_0^{r}), (\mathcal{N}_j, m_0^{j})\}$, $P_a$, $P_r$, set of activities $\mathcal{A}$, 
		set of resources $\mathcal{R}$.}
	\KwOut{Function $\rho$ and partial functions $d,b$.}
	\Begin{
		
		\tcc{Go over the activities, and check which of the resources is connected, per execution mode.}
		\For{$e \in E \setminus E'$}{$R(e) = \{r \in P_r \ | \ (r,e) \in F  \}$;\\}
		\For{$a \in \mathcal{A}$}{$\rho(a) = \{ R(e) \ | \ (a,e) \in F \}$;\\}
		\tcc{Retrieve the SD net that corresponds to a combination of $a$ and $R$.}

		
		
		\For{$a \in A$}{
		\For{$R \in \rho(a)$}{
		$i(a,R) = \forall i \in I^n(\exists e \in E_i((a,e) \in F))$;\\
		$d(a, R) = \{ \tau(e') \ | \ e' \in E'_{i(a,R)}  \}$;\\}}
		
			\Return $\rho, d, b$;\\	
	}

	\caption{Constructs: RTAA function, duration partial function, and demand partial function.}
	\label{ConstructRhoDur}
\end{algorithm}












\end{document}







